%------------------------------------------------------------------------------%
\newcommand{\normv}{\left|\left|\myol{v}\right|\right|}
\newcommand{\TF}{T_{\mathrm{F}}}
\newcommand{\LF}{L_{\mathrm{F}}}
\newcommand{\LG}{L_{\mathrm{G}}}
\newcommand{\hNR}{h_{\mathrm{NR}}}
\section{Newtonian/Non-Relativistic Limit in a Schwarzchild Spacetime
with Isotropic Coordinates}

\subsection{Preliminaries}

Under a 3+1 decomposition with a conformally-flat spatial three-metric,
the lapse function and conformal factor of the Schwarzchild spacetime expressed
in isotropic coordinates are given by \citep{bs2010}
\begin{equation}
  \alpha\left(r\right)
  :=\left(1-\frac{M}{2\,r}\right)
    \left(1+\frac{M}{2\,r}\right)^{-1}
  =\left(1+\frac{1}{2}\,\Phi\left(r\right)\right)
   \left(1-\frac{1}{2}\,\Phi\left(r\right)\right)^{-1}
\end{equation}
and
\begin{equation}
  \psi\left(r\right)
  :=1+\frac{M}{2\,r}
  =1-\frac{1}{2}\,\Phi\left(r\right),
  \label{eq.cf}
\end{equation}
where
\begin{equation}
  \Phi\left(r\right)
  :=-\frac{M}{r}
\end{equation}
is the Newtonian gravitational potential.\vspace{1em}

The Newtonian limit is defined as the limit where $\Phi\ll1$.
In this limit, the lapse function reduces to
\begin{equation}
  \alpha\approx1+\Phi,
\end{equation}
while the conformal factor remains
\begin{equation}
  \psi=1-\frac{1}{2}\,\Phi.
\end{equation}

\subsection{General Results}

The 3+1 hydrodynamics equations with a stationary spacetime can be written as
\citep{rz2013}
\begin{equation}
  \p_{t}\,\bs{U}
  +\frac{1}{\sqrt{\gamma}}\,
  \p_{i}\left(\alpha\,\sqrt{\gamma}\,\bs{F}^{i}\left(\bs{U}\right)\right)
  =\bs{S}\left(\bs{U}\right).
\end{equation}
First, we focus on the flux term:
\begin{align}
  &\frac{1}{\sqrt{\gamma}}\,
  \p_{i}\left(\alpha\,\sqrt{\gamma}\,\bs{F}^{i}\left(\bs{U}\right)\right) \\
  &=\frac{1}{\psi^{6}\,\sqrt{\gammabar}}\,
  \p_{i}\left(\alpha\,\sqrt{\gamma}\,\bs{F}^{i}\left(\bs{U}\right)\right) \\
  &\approx\frac{1+3\,\Phi}{\sqrt{\gammabar}}\,
  \p_{i}\left(\alpha\,\sqrt{\gamma}\,\bs{F}^{i}\left(\bs{U}\right)\right) \\
  &\approx\frac{1}{\sqrt{\gammabar}}\,
  \p_{i}\left(\alpha\,\sqrt{\gamma}\,\bs{F}^{i}\left(\bs{U}\right)\right) \\
  &\approx\frac{1}{\sqrt{\gammabar}}\,
  \p_{i}\left(\left(1+\Phi\right)\left(1-3\,\Phi\right)\,\sqrt{\gammabar}\,
  \bs{F}^{i}\left(\bs{U}\right)\right) \\
  &=\frac{1}{\sqrt{\gammabar}}\,
  \p_{i}\left(\left(1-2\,\Phi-3\,\Phi^{2}\right)\sqrt{\gammabar}\,
  \bs{F}^{i}\left(\bs{U}\right)\right) \\
  &=\frac{1}{\sqrt{\gammabar}}\left[
  \p_{i}\left(\sqrt{\gammabar}\,
    \bs{F}^{i}\left(\bs{U}\right)\right)
  -2\,\p_{i}\left(\Phi\,\sqrt{\gammabar}\,
    \bs{F}^{i}\left(\bs{U}\right)\right)
  -3\,\p_{i}\left(\Phi^{2}\,\sqrt{\gammabar}\,
    \bs{F}^{i}\left(\bs{U}\right)\right)\right] \\
  &\approx\frac{1}{\sqrt{\gammabar}}\left[
  \p_{i}\left(\sqrt{\gammabar}\,
    \bs{F}^{i}\left(\bs{U}\right)\right)
  -2\,\left(\p_{i}\,\Phi\right)
  \sqrt{\gammabar}\,\bs{F}^{i}\left(\bs{U}\right)\right] \\
  &=\frac{1}{\sqrt{\gammabar}}\,
  \p_{i}\left(\sqrt{\gammabar}\,
    \bs{F}^{i}\left(\bs{U}\right)\right)
  -2\,\left(\p_{i}\,\Phi\right)\,
  \bs{F}^{i}\left(\bs{U}\right).
\end{align}
The last term can be dropped by considering length scales.
We can heuristically write the expression as
\begin{equation}
  \frac{\bs{F}^{i}\left(\bs{U}\right)}{\LF}
  +\frac{\Phi}{\LG}\,\bs{F}^{i}\left(\bs{U}\right)
  =\frac{\bs{F}^{i}\left(\bs{U}\right)}{\LF}
  \left(1+\frac{\LF}{\LG}\,\Phi\right),
\end{equation}
where $\LF$ and $\LG$ are characteristic length scales, with
$\LF$ corresponding to a fluid timescale and $\LG$ to
a gravitational timescale.
In general, because gravity is a long-range force, we will have
$\LG\gg\LF$.
With that, along with $\phi\ll1$, we have
\begin{equation}
  \p_{t}\,\bs{U}
  +\frac{1}{\sqrt{\gammabar}}\,
  \p_{i}\left(\sqrt{\gammabar}\,\bs{F}^{i}\left(\bs{U}\right)\right)
  \approx\bs{S}\left(\bs{U}\right).
\end{equation}
Since one flat-space metric is as good as any other,
we assume Cartesian coordinates for the remainder of this derivation.

\subsection{Mass Equation}

For the mass equation, we have
\begin{equation}
  \p_{t}\,D+\p_{i}\,F^{i}_{D}\approx0,
\end{equation}
where
\begin{equation}
  D:=\rho\,W,
\end{equation}
\begin{equation}
  F^{i}_{D}:=\rho\,W\,v^{i},
\end{equation}
and where $W$ is the Lorentz factor,
\begin{equation}
  W:=\left(1-v^{k}v_{k}\right)^{-1/2}.
\end{equation}
The gradient of $W$ is
\begin{equation}
  \p_{\mu}\,W
  =\p_{\mu}\left(1-v^{k}v_{k}\right)^{-1/2}
  =W^{3}\,v_{k}\,\p_{\mu}\,v^{k},
\end{equation}
and the non-relativistic limit of $W$ is
\begin{equation}
  W\approx1+\frac{1}{2}\,v^{k}v_{k}\sim1+\normv^{2},
  \ \normv:=\left|\myol{v}\cdot\myol{v}\right|^{1/2}.
\end{equation}
Therefore, as far as scaling arguments go,
\begin{align}
  \p_{t}\,W&\sim\frac{\normv^{2}}{\TF}\left(1+\normv^{2}\right) \\
  \p_{i}\,W&\sim\frac{\normv^{2}}{\LF}\left(1+\normv^{2}\right).
\end{align}

\subsubsection{Time-Derivative Term}
First, we focus on the time-derivative term, for which we find
\begin{align}
  \p_{t}\left(\rho\,W\right)
  &=\left(\p_{t}\,\rho\right)W+\rho\,\p_{t}\,W \\
  &\sim\frac{\rho}{\TF}\left(1+\normv^{2}\right)
  +\rho\,\frac{\normv^{2}}{\TF}\left(1+\normv^{2}\right) \\
  &\sim\frac{\rho}{\TF}\left(1+\normv^{2}+\normv^{4}\right) \\
  &\sim\frac{\rho}{\TF} \\
  &\approx\p_{t}\,\rho.
  \label{NL.eq.rhoT}
\end{align}

\subsubsection{Flux Term}
Next, we focus on the flux term.
We note that the derivation would proceed in the same manner as that for
\eqref{NL.eq.rhoT}, except in this case we have an additional velocity
and also deal with $\LF$ instead of $\TF$.
With that, we can immediately write
\begin{align}
  \p_{i}\left(\rho\,W\,v^{i}\right)
  &\sim\frac{\rho\,\normv}{\LF}\approx\p_{i}\left(\rho\,v^{i}\right).
\end{align}
With that, we arrive at
\begin{equation}
  \p_{i}\left(\rho\,W\,v^{i}\right)
  \approx\p_{i}\left(\rho\,v^{i}\right).
\end{equation}
Combining expressions, we have our desired result,
\begin{equation}
  \p_{t}\,\rho+\p_{i}\left(\rho\,v^{i}\right)\approx0.
\end{equation}

\subsection{Momentum Equation}

For the momentum equation, we have
\begin{equation}
  \p_{t}\,S_{j}+\p_{i}\,F^{i}_{S_{j}}=S_{S_{j}},
\end{equation}
where
\begin{equation}
  S_{j}:=\rho\,h\,W^{2}\,v_{j},
\end{equation}
\begin{equation}
  F^{i}_{S_{j}}
  :=\rho\,h\,W^{2}\,v^{i}\,v_{j}+p\,\delta^{i}_{~j},
\end{equation}
and
\begin{equation}
  S_{S_{j}}
  :=\frac{1}{2}\,\alpha
  \left(\rho\,h\,W^{2}\,v^{i}\,v^{k}+p\,\gamma^{ik}\right)\,
  \p_{j}\,\gamma_{ik}-\left(\rho\,h\,W^{2}-p\right)\p_{j}\,\alpha.
\end{equation}
Here,
\begin{equation}
  h:=1+\frac{e+p}{\rho}=1+\hNR,
\end{equation}
where
\begin{equation}
  \hNR:=\frac{e+p}{\rho}.
\end{equation}
In the non-relativistic limit,
\begin{equation}
  \hNR\ll1\implies h\sim1.
\end{equation}

\subsubsection{Time-Derivative Term}
Focusing first on the time-derivative term, we have
\begin{align}
  \p_{t}\left(\rho\,h\,W^{2}\,v_{j}\right)
  &=h\,W^{2}\,\p_{t}\left(\rho\,v_{j}\right)
  +\rho\,v_{j}\,\p_{t}\left(h\,W^{2}\right) \\
  &=h\,W^{2}\,\p_{t}\left(\rho\,v_{j}\right)
  +\rho\,v_{j}\left[\left(\p_{t}\,h\right)W^{2}+2\,h\,W\,\p_{t}\,W\right] \\
  &\sim h\left(1+\normv^{2}\right)\frac{\rho\,\normv}{\TF}
  +\rho\,\normv\frac{\hNR}{\TF}\left(1+\normv^{2}\right) \notag \\
  &\hspace{2em}+\rho\,\normv\,h\left(1+\normv^{2}\right)
  \frac{\normv^{2}}{\TF}\left(1+\normv^{2}\right) \\
  &\sim\frac{\rho\,\normv}{\TF}\left(1+\normv^{2}\right)
  \left[h+\hNR+h\,\normv^{2}\left(1+\normv^{2}\right)\right] \\
  &\sim\frac{\rho\,\normv}{\TF}\left(1+\normv^{2}\right)
  \left(h+\hNR+h\,\normv^{2}+h\,\normv^{4}\right) \\
  &\sim\frac{\rho\,\normv}{\TF}\left(1+\normv^{2}\right)
  \left(1+\hNR+\normv^{2}+\hNR\,\normv^{2}\right. \notag \\
  &\left.\hspace{12em}+\normv^{4}+\hNR\,\normv^{4}\right) \\
  &\sim\frac{\rho\,\normv}{\TF} \\
  &\approx\p_{t}\left(\rho\,v_{j}\right).
  \label{NL.eq.SjT}
\end{align}

\subsubsection{Flux Term}
Next we look at the flux term.
As it did for the flux term in the mass equation,
the derivation here will proceed in the same manner as that
for \eqref{NL.eq.SjT},
except for an additional velocity and the pressure term.
With that, we have immediately
\begin{equation}
  \p_{i}\left(\rho\,h\,W^{2}\,v^{i}\,v_{j}+p\,\delta^{i}_{~j}\right)
  \approx\p_{i}\left(\rho\,v^{i}\,v_{j}+p\,\delta^{i}_{~j}\right).
\end{equation}

\subsubsection{Source Term}
Finally, we look at the sourcee term.
We have
\begin{align}
  S_{S_{j}}
  &=\frac{1}{2}\,\alpha
  \left(\rho\,h\,W^{2}\,v^{i}\,v^{k}+p\,\gamma^{ik}\right)\,
  \p_{j}\,\gamma_{ik}-\left(\rho\,h\,W^{2}-p\right)\p_{j}\,\alpha \\
  &\approx\frac{1}{2}\,\alpha
  \left(\rho\,v^{i}\,v^{k}+p\,\gamma^{ik}\right)\,
  \p_{j}\,\gamma_{ik}-\rho\,\p_{j}\,\alpha \\
  &\approx\frac{1}{2}\left(1+\Phi\right)
  \left(\rho\,v^{i}\,v^{k}+p\,\gamma^{ik}\right)\,
  \p_{j}\,\gamma_{ik}-\rho\,\p_{j}\left(1+\Phi\right) \\
  &\approx\frac{1}{2}\,
  \left(\rho\,v^{i}\,v^{k}+p\,\gamma^{ik}\right)\,
  \p_{j}\,\gamma_{ik}-\rho\,\p_{j}\,\Phi \\
  &\approx\frac{1}{2}\,
  \left(\rho\,v^{i}\,v^{k}
  +p\left(1+2\,\Phi\right)\gammabar^{ik}\right)\,
  \p_{j}\left[\left(1-2\,\Phi\right)\gammabar_{ik}\right]
  -\rho\,\p_{j}\,\Phi \\
  &\approx\frac{1}{2}\,
  \left(\rho\,v^{i}\,v^{k}
  +p\,\gammabar^{ik}\right)\,
  \p_{j}\left[\left(1-2\,\Phi\right)\gammabar_{ik}\right]
  -\rho\,\p_{j}\,\Phi \\
  &\approx\frac{1}{2}\,
  \left(\rho\,v^{i}\,v^{k}
  +p\,\gammabar^{ik}\right)
  \left[\p_{j}\,\gammabar_{ik}
  -2\,\p_{j}\left(\Phi\,\gammabar_{ik}\right)\right]
  -\rho\,\p_{j}\,\Phi \\
  &\approx\frac{1}{2}\,
  \left(\rho\,v^{i}\,v^{k}
  +p\,\gammabar^{ik}\right)
  \left[\p_{j}\,\gammabar_{ik}
  -2\,\gammabar_{ik}\,\p_{j}\,\Phi
  -2\,\Phi\,\p_{j}\,\gammabar_{ik}\right]
  -\rho\,\p_{j}\,\Phi \\
  &\approx\frac{1}{2}\,
  \left(\rho\,v^{i}\,v^{k}
  +p\,\gammabar^{ik}\right)
  \left[\p_{j}\,\gammabar_{ik}
  -2\,\gammabar_{ik}\,\p_{j}\,\Phi\right]
  -\rho\,\p_{j}\,\Phi \\
  &=\frac{1}{2}\,
  \left(\rho\,v^{i}\,v^{k}
  +p\,\gammabar^{ik}\right)
  \left[\p_{j}-2\,\p_{j}\,\Phi\right]\gammabar_{ik}
  -\rho\,\p_{j}\,\Phi.
\end{align}
Focusing on the operator in square brackets, we have
\begin{equation}
  \frac{1}{\LG}+\frac{\Phi}{\LG}
  =\frac{1}{\LG}\left(1+\Phi\right)
  \approx\frac{1}{\LG}.
\end{equation}
With that, we have our result,
\begin{equation}
  \p_{t}\left(\rho\,v_{j}\right)
  +\p_{i}\left(\rho\,v^{i}\,v_{j}+p\,\delta^{i}_{~j}\right)
  \approx\frac{1}{2}\left(\rho\,v^{i}\,v^{k}+p\,\gammabar^{ik}\right)
  \p_{j}\,\gammabar_{ik}-\rho\,\p_{j}\,\Phi.
\end{equation}

\subsection{Energy Equation}

The energy equation is
\begin{equation}
  \p_{t}\,\tau+\p_{i}\,F^{i}_{\tau}=S_{\tau},
\end{equation}
where
\begin{equation}
  \tau:=\rho\,h\,W^{2}-p-\rho\,W,
\end{equation}
\begin{equation}
  F^{i}_{\tau}:=\rho\,h\,W^{2}\,v^{i}-\rho\,W\,v^{i},
\end{equation}
and
\begin{equation}
  S_{\tau}:=-\rho\,h\,W^{2}\,v^{j}\,\p_{j}\,\alpha.
\end{equation}

\subsubsection{Time-Derivative Term}

We have
\begin{align}
  \tau&=\rho\,h\,W^{2}-p-\rho\,W \\
  &\approx\rho\left(1+\hNR\right)\left(1+v^{k}v_{k}\right)
  -p-\rho\left(1+\frac{1}{2}\,v^{k}v_{k}\right) \\
  &\approx\rho\left(1+\hNR+v^{k}v_{k}\right)
  -p-\rho-\frac{1}{2}\,\rho\,v^{k}v_{k} \\
  &=e+\frac{1}{2}\,\rho\,v^{k}v_{k},
\end{align}
leading to
\begin{equation}
  \p_{t}\,\tau\approx\p_{t}\left(e+\frac{1}{2}\,\rho\,v^{k}v_{k}\right).
\end{equation}

\subsubsection{Flux Term}

For the flux term,
\begin{align}
  F^{i}_{\tau}&=\rho\,h\,W^{2}\,v^{i}-\rho\,W\,v^{i} \\
  &=\rho\,h\,W^{2}\,v^{i}-\rho\,W\,v^{i}+p\,v^{i}-p\,v^{i} \\
  &=\left(\rho\,h\,W^{2}-p-\rho\,W\right)v^{i}+p\,v^{i} \\
  &=\tau\,v^{i}+p\,v^{i}
\end{align}
The same derivation as for the time-derivative term leads to
\begin{equation}
  \p_{i}\,F^{i}_{\tau}
  \approx\p_{i}\left[\left(e+\frac{1}{2}\,\rho\,v^{k}v_{k}+p\right)v^{i}\right].
\end{equation}

\subsubsection{Source Term}

For the source term,
\begin{align}
  S_{\tau}&=-\rho\,h\,W^{2}\,v^{j}\,\p_{j}\,\alpha \\
  &\approx-\rho\,h\,W^{2}\,v^{j}\,\p_{j}\left(1+\Phi\right) \\
  &=-\rho\,h\,W^{2}\,v^{j}\,\p_{j}\,\Phi \\
  &\approx-rho\,v^{j}\,\p_{j}\,\Phi.
\end{align}
Putting things together,
\begin{equation}
  \p_{t}\left(e+\frac{1}{2}\,\rho\,v^{k}v_{k}\right)
  +\p_{i}\left[\left(e+\frac{1}{2}\,\rho\,v^{k}v_{k}+p\right)v^{i}\right]
  \approx-\rho\,v^{j}\,\p_{j}\,\Phi.
\end{equation}

%------------------------------------------------------------------------------%
\let\normv\undefined
\let\TF\undefined
\let\LF\undefined
\let\LG\undefined
\let\hNR\undefined
