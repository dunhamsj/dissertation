\section{Vector Components}

A generic four-vector, $u$, can be expanded in any coordinate basis
(e.g., the 3+1 coordinate basis),
$\left\{\p/\p x^{\mu}\right\}$, as
\begin{equation}
  u=u_{\left(x\right)}^{\mu}\,\pd{}{x^{\mu}},
\end{equation}
where $\p/\p x^{\mu}$ is the $\mu$-th basis vector
and $u_{\left(x\right)}^{\mu}$ is the $\mu$-th component of $u$ in
that basis.
In general, these basis vectors are not orthonormal;
i.e., $g\left(\p/\p x^{\mu},\p/\p x^{\nu}\right)
=:g^{\left(x\right)}_{\mu\nu}\neq\eta_{\mu\nu}$,
where, here, $g$ is the $\left(0,2\right)$ metric tensor.
We can also expand $u$ in an orthonormal basis, $\left\{e_{\mu}\right\}$,
\begin{equation}
  u=u^{\mu}\,e_{\mu},
\end{equation}
where the $e_{\mu}$ are defined such that
$g\left(e_{\mu},e_{\nu}\right)=\eta_{\mu\nu}$.
If $e_{0}$ is a four-velocity then this basis can be associated
with an observer, and $\left\{e_{\mu}\right\}$ can properly
be called a frame of reference.
In particular, if $e_{0}\equiv n$ (see \secref{ss.3+1}) then this frame defines
an Eulerian observer, and $\left\{e_{\mu}\right\}$ is the frame of
reference of an Eulerian observer.

As an example of immediate relevance,
the four-velocity of the fluid as measured by an Eulerian observer
in the 3+1 basis (in which the 0-th component is zero) is
\begin{equation}
  v=v_{\left(x\right)}^{\mu}\,\pd{}{x^{\mu}}
  =v_{\left(x\right)}^{i}\,\pd{}{x^{i}}.
\end{equation}
The $v_{\left(x\right)}^{i}$
are what we initialize in our test problems and applications.
In spherical-polar coordinates, this vector takes the form
\begin{equation}
  v=v_{\left(x\right)}^{r}\,\pd{}{r}
  +v_{\left(x\right)}^{\theta}\,\pd{}{\theta}
  +v_{\left(x\right)}^{\varphi}\,\pd{}{\varphi}.
\end{equation}
This should be contrasted with the perhaps more familiar form,
\begin{equation}
  v=v^{i}\,e_{i}
  =v^{r}\,e_{r}+v^{\theta}\,e_{\theta}
  +v^{\varphi}\,e_{\varphi},
\end{equation}
where
\begin{equation}
  e_{i}:=\left|\left|\pd{}{x^{i}}\right|\right|^{-1}\,
  \pd{}{x^{i}}\hspace{1em}\left(\mathrm{no\ sum\ over\ }i\right),
\end{equation}
with
\begin{equation}
  \left|\left|\pd{}{x^{i}}\right|\right|
  :=\left|g\left(\pd{}{x^{i}},
  \pd{}{x^{i}}\right)\right|^{1/2}
  =\left|g^{\left(x\right)}_{ii}\right|^{1/2}=h^{\left(x\right)}_{i},
\end{equation}
and where $h^{\left(x\right)}_{i}$ is the $i$-th scale factor in the
coordinate system $\left\{\p/\p x^{\mu}\right\}$.
From this, we see that the components of $v$ in the 3+1 basis,
$v_{\left(x\right)}^{i}$,
are related to the components of $v$ in the basis of the Eulerian
observer, $v^{i}$, as
\begin{equation}
  v^{i}
  =h^{\left(x\right)}_{i}\,v_{\left(x\right)}^{i}
  \ \left(\mathrm{no\ sum\ over\ }i\right).
\end{equation}

Also, note that the $v_{\left(x\right)}^{i}$ are different from
the coordinate velocities,
which are instead defined as (dropping the $\left(x\right)$ subscript)
\begin{equation}
  \dot{x}^{i}:=\td{x^{i}}{t}
  =\td{\tau}{t}\,\td{x^{i}}{\tau}
  =\frac{u^{i}}{u^{0}}
  =\frac{W}{u^{0}}\left(v^{i}+n^{i}\right)
  =\alpha\left(v^{i}+n^{i}\right)
  =\alpha\,v^{i}-\beta^{i}.
\end{equation}
