

\chapter{Title of Chapter: Manuscript Preparation} \label{CH1_Introduction}
%\vspace{-7mm}
%\bigskip
These guidelines provide students at Vanderbilt University with essential information about how to prepare and submit theses and dissertations in a format acceptable to the Graduate School. The topics range from writing style to the completion of required forms. There are instructions and sample pages on the Graduate School website for guidance through this process.


\section{Style}
There is a distinct difference between submitting a manuscript to a publisher and providing a completed thesis or dissertation to the Graduate School. A manuscript represents a pre-publication format; a thesis or dissertation is a final, completely edited, published document. Students should use these guidelines, not other style manuals, as the final authority on issues of format and style. Areas not covered in this document or deviation from any of the specifications should be discussed with a Graduate School format editor. Do not use previously accepted theses and dissertations as definite models for style.

\section{Composition and Structure}
Manuscripts consist of four major sections and must be placed in the order listed:


\subsection{Preliminary Pages}
Title Page (required) \\
Copyright (optional, Ph.D. only) \\
Dedication (optional)  \\
Acknowledgment(optional) \\
Preface (optional) \\
Table of Contents (required) \\ 
List of Tables (required) \\
List of Figures(required) \\
List of Abbreviations/Nomenclature/Symbols (optional)


\subsection{Text}
Introduction (may be referred to as Chapter 1) \\
Body of Manuscript


\subsection{References}
References (required)

\subsection{Appendices}
Appendices (optional)


\section{Acknowledgment of Support}
The acknowledgements page is optional. Acknowledgement of grant and contract support may be included on this page. This is also where you thank the people that made your work possible: advisers, your committee, mom, friends,  etc.

\section{Abstract}
The abstract is a separate document from the manuscript; it is not bound with the thesis or dissertation. Abstracts must be printed on white, $8\frac{1}{2} \times 11$-inch paper. No page numbers are printed on the abstract. One copy is required. Abstracts must have the original signature(s) of the faculty advisor(s). The maximum length of the thesis abstract is 250 words. The maximum length of the dissertation abstract is 350 words, including the dissertation title. Majors are listed on the last pages of these guidelines. 

\section{Title Page}
The title page must be printed on white, $8\frac{1}{2} \times 11$-inch paper. Committee member signatures on the title page must be originals. Spacing on the title page will vary according to the length of the title. The five lines following your name must be formatted exactly as found on the sample title page. The title page is considered page ‘i’ but the page number is not printed on the page. \textbf{The \underline{month}, \underline{day}, and \underline{year} representing the conferral date must be listed on the title page}. 

\section{Font}
Use a standard font consistently throughout the manuscript. Font size should be 10 to 12 point for all text, including titles and headings. It is permissible to change point size in tables, figures, captions, footnotes, and appendix material. Retain the same font, where possible. When charts, graphs, or spreadsheets are “imported,” it is permissible to use alternate fonts. Italics are appropriate for book and journal titles, foreign terms, and scientific terminology. \textbf{Boldface} may be used within the text for emphasis and/or for headings and subheadings. Use both in moderation. 

\section{Margins}
Measure the top margin from the edge of the page to the top of the first line of text. Measure the bottom page margin from the bottom of the last line of text to the bottom edge of the page. Page margins should be a minimum of one-half inch from top, bottom, left and right and a maximum of one inch from top, bottom, left and right. Right margins may be justified or ragged, depending upon departmental requirements or student preference. 

\section{Pagination}
The title page is considered to be page one, but the page number should not be printed on this page. All other pages should have a page number centered about ½ inch from the bottom of the page. Number the preliminary pages in lowercase Roman numerals. Arabic numerals begin on the first page of 3 text. Pages are numbered consecutively throughout the remainder of the manuscript. The Introduction may be placed before the first page of Chapter 1, if it is not considered a chapter. The use of Arabic numbers may begin on the first page of the Introduction. 


\section{Spacing}
The entire text may be single-spaced, one and one-half spaced, or double-spaced. Block quotations, footnotes, endnotes, table and figure captions, titles longer than one line, and individual reference entries may be single-spaced. With spacing set, the following guidelines should be applied: Two enters after chapter numbers, chapter titles and major section titles (Dedication, Acknowledgements, Table of Contents, List of Tables, List of Figures, List of Abbreviations, Appendices, and References). Two enters before each first- level and second-level heading. Two enters before and after tables and figures embedded in the text. One enter after sub-level headings


\section{Numbering Schemes}
Chapters may be identified with uppercase Roman numerals or Arabic numbers. Styles used on the Table of Contents should be consistent within the text. Tables, figures, footnotes, and equations should be numbered consecutively throughout the manuscript with Arabic numerals. These may also be numbered consecutively by each chapter. Equation numbers should be placed to the right of the equation and contained within parentheses or brackets. Use uppercase letters to designate appendices.



\section{Division}

\subsection{Body of Manuscript}
Departments will determine acceptable standards for organizing master’s theses into chapters, sections, or parts. Usually, if a thesis has headings, a Table of Contents should be included. The dissertation must be divided into chapters. The use of parts, in addition to chapters, is acceptable. 

\subsection{Words and Sentences}
Take care to divide words correctly. Do not divide words from one page to the next. Word processing software provides for “widow and orphan” protection. Utilize this feature to help in the proper division of sentences from one page to another. In general, a single line of text should not be left at the bottom or top of a page. Blank space may be left at the bottom of a page, where necessary.

\subsection{Headings and Subheadings}
Use headings and subheadings to briefly describe the material in the section that follows.\underline{\textbf{Be consistent}} with your choice of “levels” and refer to the instructions on spacing for proper spacing between headings, subheadings, and text. First-level headings must be listed on the Table of Contents. Second-level and subsequent subheadings may be included.




\subsection{Acronyms/Abbreviations/Capitalization}
Abbreviations on the title page should appear as they do in the body of the thesis or dissertation. (Examples: Xenopus laevis, Ca, Mg, Pb, Zn; TGF-$\beta$, p53.) Capitalize only the first letter of words of importance, distinction, or emphasis in titles and headings. Do not alter the all-cap style used for acronyms (Example: AIDS) and organizational names (Example: IBM). Use the conventional style for Latin words (Examples: in vitro, in vivo, in situ). Genus and species should be italicized. Capitalize the first letter of the genus, but not that of the species name (Example: Streptococcus aureus).


\section{Tables and Figures}
Figures commonly refer to photographs, images, maps, charts, graphs, and drawings. Tables generally list tabulated numerical data. These items should appear as close as possible to their first mention in the text. Tables and figures may be placed in appendices if this is a departmental requirement or standard in the field. Tables and figures should be numbered with Arabic numerals, either consecutively or by chapter. \underline{\textbf{Be consistent}} in the style used in the placement of tables and figure captions. Tables and figures may be embedded within the text or placed on a page alone. When placed on its own page, a figure or table may be centered on the page. When included with text, a table or figure should be set apart from the text. Tables and figures, including captions, may be oriented in landscape. Make sure to use landscape page positioning on landscape-oriented pages. Table data and figure data must be kept together if the information fits on one page.












