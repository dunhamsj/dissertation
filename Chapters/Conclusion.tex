\chapter{Summary/Conclusion/Future Work\ee{Summary, Conclusions, and Outlook?}}

We have shown, via detailed numerical simulations, that GR may play an
important role in \ee{our further understanding of} the SASI, and that for simulation codes to
accurately predict observables, they should use GR gravity and GR hydrodynamics.
In particular, in order to correctly capture the growth rate of the
instability, GR must be used.  
This is important for the creation of templates for gravitational waves
and may also be important for CCSNe with more massive progenitors.
\ee{What are the limitations of your work, and what should be done next?  You are now the expert and should provide your perspective on this.}

We have detailed our implementation of the DG method to solve the GRHD
system of equations.\ee{It is not just the GRHD system, but the ``Euler--Einstein'' system.}
This includes how we mitigate spurious oscillations,
enforce physical bounds, and update the solution in time.
Additionally, we discussed how we coupled \thornado\ to \poseidon,
a gravity solver for the \xcfc\ system of equations.
We have also detailed our coupling to \amrex, including how we interpolate
fluid fields across different levels of refinement in a conservative manner
and how we achieve conservation across coarse-to-fine interfaces.

We have verified and validated \ee{verification and validation has a specific meaning: one V (I forget which one) is about making sure you solve the proposed model correctly, the other V is making sure your model is the correct one for the system you want to study.} our implementation by performing several
simulations of test-problems as well as by simulating the adiabatic collapse,
bounce, and
post-bounce evolution of a 15 Solar mass progenitor in parallel, with AMR.
We have shown that DG methods are appropriate for this application \ee{how?},
and future work involves coupling the existing framework to the DG, GR
neutrino transport module and performing full, three-dimensional
CCSN simulations.
\ee{What other challenges remain?  How do you propose to deal with CFL condition in 2D/3D when using spherical-polar coordinates? Discuss some computational challenges associated with neutrino transport; e.g., more degrees of freedom, local physics that can lead to load imbalance.}

Future work also includes finishing the methods paper (Chapter~\ref{ch.mp}),
which will involve
simulating a SASI as we have done in \citet{dem2023}, but using AMR.
We will compute the power in the $\ell=1$ mode for simulations performed
with and without AMR and examine the results.
Future work will also involve comparing the AMR adiabatic collapse simulations
to determine what resolution is sufficient to maintain stability of the PNS.
\ee{This is ``future work'' for finishing your thesis.  You should provide some perspectives on the science that can be done with the tool you have built.  Your committee will likely want to hear about that.}

Beyond finishing the methods paper, future work involves porting the
\thornado-\amrex\ coupling to GPUs,
a necessity for taking full advantage of current, and future,
leadership-class supercomputers.
