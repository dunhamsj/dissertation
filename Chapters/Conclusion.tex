\chapter{Summary, Conclusions, and Outlook}
\label{ch.con}

We have shown, via detailed numerical simulations, that GR may play an
important role in our further understanding of the SASI, and that for simulation codes to
accurately predict observables, they should use GR gravity and GR hydrodynamics.
In particular, in order to correctly capture the growth rate of the
instability, GR must be used.
This is important for the creation of templates for gravitational waves
and may also be important for CCSNe with more massive progenitors.
As mentioned in the conclusion of Chapter~\ref{ch.sasi},
next steps for this project include performing 3D simulations.
The behavior of the additional spiral modes may also influenced by GR,
and only simulations will reveal that.
The simulations we have performed revealed aspects of the pure SASI and its
behavior under GR;
it would also be worthwhile to perform simulations with some implementation of
neutrino transport because the SASI in CCSNe may behave differently than
the SASI we have investigated.
This would be difficult because disentangling neutrino-driven convection
from the SASI would be a non-trivial task.

We have detailed our implementation of the DG method to solve the GRHD
``Euler--Einstein" system of equations.
This includes how we mitigate spurious oscillations,
enforce physical bounds, and update the solution in time.
Additionally, we discussed how we coupled \thornado\ to \poseidon,
a gravity solver for the \xcfc\ system of equations.
We have also detailed our coupling to \amrex, including how we interpolate
fluid fields across different levels of refinement in a conservative manner
and how we achieve conservation across coarse-to-fine interfaces.
We have checked our implementation by performing several
simulations of test-problems as well as by simulating the adiabatic collapse,
bounce, and
post-bounce evolution of a 15 Solar mass progenitor in parallel, with AMR.
Future work includes performing the same simulation but in 2D and 3D.
This will take some development because of the restriction on the timestep
imposed at the origin and at the poles;
one possible remedy is to implement a filtering scheme as is done in
\texttt{SphericalNR} \citep{jmz2023}.
Future work also includes
simulating a SASI as we have done in \citet{dem2023}, but using AMR.
We will compute the power in the $\ell=1$ mode for simulations performed
with and without AMR and examine the results.
Future work will also involve comparing the AMR adiabatic collapse simulations
to determine what resolution is sufficient to maintain stability of the PNS.

Beyond finishing the methods paper, future work involves porting the
\thornado-\amrex\ coupling to GPUs,
a necessity for taking full advantage of current, and future,
leadership-class supercomputers.
Future work also includes coupling the existing framework to the DG, GR
neutrino transport module and performing full, three-dimensional
CCSN simulations.
This will bring its own challenges, particularly in multi-dimensional simulations
because of the large number of degrees of freedom imposed by a spectral,
multi-group two-moment scheme.

In the meantime, we have a code that solves the GRHD equations in AMR coupled
to an \xcfc\ metric solver.
This code is free and open-source and available to anyone who wishes to use it
for their application.
For example, the code could potentially be used to simulate the late stages
of a CCSN progenitor, when the neutrinos are all in the free-streaming regime.
