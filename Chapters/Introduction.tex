\chapter{Introduction}

Core-collapse supernovae are sources of some of the most extraordinary and
exquisite objects in the Universe, ranging from the microscopic heavy elements
they synthesize to the macroscopic neutron stars/pulsars in
which they often result.
In addition, they are unique laboratories in which we can probe areas of
physics which are otherwise difficult, if not impossible, to probe with
terrestrial-based experiments; e.g. nuclear matter and neutrinos.
During a core-collapse supernova (CCSN), the progenitor releases its immense
gravitational potential energy ($\sim10^{46}$ J), of which a mere 1\% goes into
the spectacular luminous displays we observe, while the remaining 99\% goes
into the dramatic release of neutrinos and antineutrinos of all flavors.
However, the gory details of exactly how a CCSN explodes are yet to be resolved.

Due in part to the fact that supernovae can outshine their entire host galaxy,
they have been observed for centuries, with some even visible during the
daytime!
However, it wasn't until the early 20th century that anyone had any ideas about
what they were \citep{bw2017}.
By studying the spectra of supernovae (SNe) it was discovered that there were,
broadly speaking, two distinct types: those whose spectra contained hydrogen
lines, and those whose spectra did not contain hydrogen lines.
Those whose spectra did not contain hydrogen lines were labeled as Type I,
and those whose spectra did contain hydrogen lines were labeled as Type II.
As more spectra were studied, it was found that there were subclasses of these
two types.
For example, a Type IIP supernova shows a plateau in its light curve,
whereas other SNe do not.
One particular subclass of Type I SNe, Type Ia, are thought to be the result of
white dwarves accreting matter to the point at which they reach their effective
Chandrasekhar masses and undergo thermonuclear explosions.
While these types of SNe are important,
but they are not the focus of this dissertation.
The Type II SNe, along with the Types Ib and Type Ic SNe, are thought to be
{\it core-collapse supernovae}.
These result, as the name implies, from the collapse of the core of a
massive star.
This results in a tremendous explosion, expelling much of the outer-layers of
the star, while keeping many heavy nuclei bound-up in the core
of the newly formed {\it proto-neutron star} (PNS).

The light-curves of SNe vary from type to type, but they do share a general
trend: they rise to a maximum on the order of days
(the so-called {\it pre-maximum phase}) and then they drop, nearly linearly in
magnitudes (exponentially in flux), for weeks.
The period of maximum light is thought to be the moment that the shock reaches
the photosphere of the stellar envelope \citep{bw2017}.

The spectra of SNe also provide a wealth of information.
Spectra are the only means we have to ascertain the composition of a SN as a
function of physical depth within the SN.
This is because as the SN ejecta expand, the optical depth lessens,
exposing deeper layers of the star \citep{bw2017}.
This is one way in which the distributions of heavy elements by CCSNe can be
deduced.

In order to understand and appreciate the products of CCSNe we need to
understand how they operate.
The following paragraph is a paraphrase from the review article \citet{m2005}.
The basic picture is that when a massive star
(ZAMS mass $M\gtrsim8\,M_{\odot}$) runs out of fuel, gravity overcomes pressure
and the star begins to collapse.
The collapse of the central iron-core separates into two pieces: a subsonically
collapsing ``inner-core" and a supersonically collapsing ``outer-core".
The collapse of the inner-core is eventually stopped because the nuclei are
forced into such close proximity with each other that they undergo a
phase transition from individual nuclei to bulk nuclear matter
(the creation of a PNS),
causing the pressure to increase drastically as a result of the repulsive
nature of the strong force.
At this point the core is supported by a combination of the strong force and
neutron degeneracy pressure.
The still infalling outer-core collides suddenly with the newly formed PNS,
generating a strong shock wave that propagates outward, ultimately driving
the explosion.
It is intuitive that the shock wave simply propagates outward, disrupting
the star.
Unfortunately, as often happens, reality is not so simple.

As the shock propagates, it loses energy to the newly-created and liberated
(anti)neutrinos as well as to the dissociation of the
infalling material through which it passes,
so much so that the shock stalls around 200 km from the center
(e.g., see \citet{hm1981}).
It is clear that the shock is revived, because if it weren't then matter
would continue to accrete onto the star, eventually making the core massive
enough to overcome the strong force and the neutron degeneracy pressure
and cause the PNS to collapse
to a black hole \citep{bw1985}.
(This does happen under certain conditions, but they are not discussed here.)

It was discovered numerically that while the shock is stable to
perturbations in 1D models, it is unstable to non-radial perturbations
in 2D models \citep{bmd2003}.
This purely hydrodynamical instability has since become known as the
standing accretion shock instability (SASI), and it very likely plays an
important role in the explosion mechanism for more massive progenitors.
This discovery also strongly implies that supernova explosions are
inherently multi-dimensional.

One effect of the SASI is turbulence generated below the shock,
which can increase the amount of time that a fluid particle spends
in the ``gain region", a region below the shock in which there is a
net heating due to neutrinos \citep{co2015}.
Since the fluid spends more time in this region it can be heated more than if
SASI-driven turbulence were absent.
This extra heating can help revive the shock.
Multi-dimensional models are necessary for several reasons,
one of which is that turbulence is non-existent in 1D, and in 2D it
exhibits an inverse-cascade, where energy is artificially transmitted
from small-scale eddies to large-scale eddies \citep{yem2017}.
Therefore, to truly study turbulence in CCSNe we need 3D simulations,
with no assumptions of symmetry.

Although exoergic fusion stops at iron,
CCSNe produce elements of much higher atomic mass number, $A$.
In fact, CCSNe are responsible for most of the elements in the Universe with
$16\lesssim A\lesssim90$ \citep{bw2017}.
Two products in particular, $^{56}$Ni and $^{57}$Ni, are critical for SN light
curves (their half-lives correlate well with the decline of the light curve),
and the amounts of these elements depend on the explosion mechanism.
The explosion energy is sufficient to expel the matter away from the star
and throughout the galaxy, enriching it with heavy elements otherwise that have
no earthly business in the ISM\footnote{Or a Maine hayfield...}.
The distribution of metals in galaxies has implications for galactic
evolution, so an understanding of how the heavy elements are distributed,
and especially how much of each element is distributed,
is crucial for galaxy evolution models.

One of the elements produced and distributed is carbon,
the element on which life as we know it is based, so there is additionally
a metaphysical reason to understand CCSNe:
It is to them that we literally owe our entire existence!

The most obvious products of CCSNe are the copious amount of photons
released in the explosion process.
However, preceding the fireworks is something arguably much more interesting
and magnificent---the gravitational waves emitted from the asymmetry of the
mass-energy distribution.
These waves will be more like bursts, lasting less than one second,
but will be in the LIGO band and should be detectable by aLIGO
out to several kpc \citep{aaa2016}.
From full simulations of CCSNe we obtain the components of the stress-energy
tensor, from which we can compute the mass-quadrupole moment,
which can tell us the waveforms of the gravitational waves emitted.
The inclusion of GR hydrodynamics will produce waveforms that are more
faithful to nature than those that currently exist.

Another product of CCSNe is the plethora of neutrinos released.
As mentioned earlier, about 99\% of the energy in a CCSN
(amounting to approximately $0.15\,\msun\,c^{2}$) is carried away by neutrinos.
These weakly interacting particles are some of the most elusive particles known.
Their practically infinitesimal cross-section makes them extremely difficult
to detect.
A CCSN is one of the few phenomena that produces a sufficient amount of
neutrinos to be detectable on Earth.

%------------------------------------------------------------------------------%
