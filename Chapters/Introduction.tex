\chapter{Introduction}

Stars more massive than about eight Solar masses end their nuclear
fusion-driven phase of existence in explosive, spectacular events
known as core-collapse supernovae (CCSNe).
These events are sources of some of the most extraordinary and
exquisite objects in the Universe, ranging from the microscopic heavy elements
they synthesize and distribute
to the macroscopic neutron stars into which their
(mostly) iron cores often transmute.
In addition, they are unique laboratories in which we can probe areas of
physics that are otherwise difficult, if not impossible, to probe with
terrestrial-based experiments; e.g., properties of dense nuclear matter and the
ever-elusive neutrino.
During a CCSN, the progenitor star releases its immense gravitational energy
($\sim\!\!10^{46}\,\J$, or $\sim\!\!10^{53}\,\erg$),
of which a mere 1\% goes into
the spectacular luminous displays we observe, while the remaining 99\% goes
into the release of neutrinos and antineutrinos \citep{bw2017}.
The violent motion of the stellar fluid generates gravitational waves that
are expected to be detectable with current generation gravitational wave
observatories such as LIGO, VIGO, and KAGRA \citep{aaa2020,aaa2020a}
and next-generation gravitational wave observatories such as
the Einstein Telescope \citep{mvb2020} and Cosmic Explorer \citep{eaa2021}.

In order to understand and appreciate the products of CCSNe, we need to
understand how they operate.
The following paragraph is a paraphrase from the review article \citet{m2005}.
The basic picture is that when a massive star
(ZAMS mass $M\gtrsim8\,\msun$, with $\msun$ the mass of the Sun)
runs out of fuel, gravity overcomes pressure
and the star begins to collapse.
As the iron-core collapses, it can be thought of as being composed of
two pieces: a subsonically
collapsing ``inner-core" and a supersonically collapsing ``outer-core".
The collapse of the inner-core is eventually stopped because the nuclei are
forced into such close proximity with each other that they undergo a
phase transition from individual nuclei to bulk nuclear matter
(this is the creation of a proto-neutron star (PNS)),
causing the pressure to increase drastically as a result of the repulsive
nature of the strong force.
At this point, the core is supported by a combination of the strong force and
neutron degeneracy pressure.
The still infalling outer-core collides suddenly with the newly formed PNS,
generating a pressure wave that develops into shock wave.
The shock wave propagates outward, ultimately driving the explosion.
It is intuitive that the shock wave simply propagates outward, disrupting
the star.
Unfortunately, as often happens, reality is not so simple.

Because of the numerous offerrings of CCSNe, efforts to model them have been
ongoing since the 1960s.
In early models, as the shock propagates,
it was seen to energy to the newly-created and liberated
electron-type neutrinos as well as to the dissociation of the
infalling material through which it passes,
so much so that the shock stalls around 200 km from the center
(e.g., see \citet{hm1981}).
It is clear that the shock is revived, because if it weren't then matter
would continue to accrete onto the star, eventually making the core massive
enough to overcome the strong force and the neutron degeneracy pressure
and cause the PNS to collapse
to a black hole \citep{bw1985}.
(This does happen under certain conditions, but they are not discussed here.)
Even as technology improved and one-dimensional (1D)
models became more sophisticated, they still did not explode
\citep[e.g., see][]{%
cj1960,
mw1966,
cw1966,
r1979,
hm1981,
bw1985,
mim1991,
mb1993,
mb1993a,
mb1993b,
rj2000,
mlm2001,
rj2002,
twi2003,
tbp2003,
lmm2004,
kjh2006,
sys2007,
zwh2008,
hmj2010,
thr2017,
sro2019,
bmo2021,
gmm2023%
}.

With further technological advances, two-dimensional (2D) models
became feasible.
This allowed for the simultaneous inflow and outflow of material,
thereby allowing for the development of neutrino-driven convection
\citep[e.g., see][]{%
hbc1992,
sys1993,
jm1995,
jm1995a,
mcb1998,
kpj2003,
kjh2006,
sjf2008,
fkh2011,
mm2011,
mjw2012,
tks2012,
mdb2013,
roa2016,
gmm2023%
}.

2D flows lead to another instability.
It was discovered numerically that while the shock is stable to
perturbations in 1D models, it is unstable to non-radial perturbations
in 2D models \citep{bmd2003}.
This purely hydrodynamical instability has since become known as the
standing accretion shock instability (SASI) and it very likely plays an
important role in the explosion mechanism for more massive progenitors.
The SASI, along with neutrino-driven convection, strongly implies that
CCSNe are inherently multi-dimensional.
One effect of the SASI is turbulence generated below the shock,
which, in addition to neutrino-driven convection,
can increase the amount of time that a fluid particle spends
in the ``gain region", a region below the shock in which there is a
net heating due to neutrinos \citep{co2015}.
Since the fluid spends more time in this region it can be heated more than if
turbulence and convection were absent.
This extra heating can help revive the shock.
Additionally, the induced Reynolds stresses can provide additional
pressure support against the shock \citep[e.g., see][]{mm2018}.

While 1D and 2D models are very useful,
three-dimensional (3D) models are necessary for several reasons,
one of which is that turbulence is non-existent in 1D and in 2D it
exhibits an inverse energy-cascade, where energy is artificially transmitted
from small scales to large scales \citep{k1967}.
Therefore, to truly study turbulence in CCSNe we need 3D simulations,
with no assumptions of symmetry.
3D CCSN simulations are computationally expensive,
taking weeks to months to complete
on leadership-class supercomputers; it is therefore
necessary to develop codes that are efficient and can scale well on
multiple CPU cores as well as on GPUs.
Several production codes are currently used to simulate CCSNe, including
\flashx\ \citep{for2000},
\cocov\ \citep{mjd2010},
\zelmani\ \citep{oao2012},
the code described in \citet{ktk2016},
\fornax\ \citep{sdb2019},
\nadafld\ \citep{rjj2019},
\chimera\ \citep{bbh2020},
and \gmunu\ \citep{cht2023}.
All of these codes use some combination of high-order finite-difference
and/or finite-volume methods.
Finite-volume methods \citep[e.g., see][]{l2002} evolve the cell average
of a particular element and use data from neighbor cells to reconstruct
a high-order polynomial representation of the solution within that element.
This allows for high-order accuracy, albeit at the cost of a wider stencil.
This is one motivation for using discontinuous Galerkin (DG) methods
(see Chapter~\ref{ch.mp} for details on the DG method):
DG methods achieve high-order accuracy on a compact (nearest-neighbors only)
stencil.
This make them appealing for executing massively parallel simulations on
supercomputing clusters.

CCSNe span orders of magnitude in space.
Because of this, it is necessary for 3D simulations to run with a computational
mesh that is adaptive in the sense that it will adjust itself such that
computational resources are concentrated in regions where they are really needed.
This so-called adaptive mesh refinement (AMR) can be implemented in different
ways.
One way is so-called ``patch-based" AMR, where chunks of the mesh, spanning
several elements in each dimension, are refined.
This is the method used by \amrex, a software framework designed to give
application codes the ability to run with AMR.
Much of my dissertation work has focused on coupling \thornado\ to \amrex\ in
order to give \thornado\ AMR capabilities (see Chapter~\ref{ch.mp} for details).

The characteristic masses of CCSN progenitors are such that general relativity
(GR) is important \citep[e.g., see][]{lmt2001}.
Although early 1D simulations included full GR
\citep[e.g.,][]{mw1966}, modern multi-dimensional simulations approximate
GR to some degree.
Specifically, \zelmani, the code described in
\citet{ktk2016}, and \nadafld\ use free evolution models;
\cocov\ and \gmunu\ use \xcfc;
and \fornax, \chimera, and \flashx\ use the effective potential.
The impact of GR has been investigated in 1D \citep{bdm2001,lmm2012}
and 2D \citep{mjm2012,oc2018}.
In 1D, \citet{bdm2001} have shown that GR leads to a more compact PNS than is
obtained with Newtonian gravity, which leads to less extended neutrinospheres
and harder neutrino spectra.
Because most neutrino-matter cross-sections relevant to CCSNe are proportional to
the square of the rms neutrino energy, harder neutrino spectra lead to more
energy being deposited into the post-shock fluid.
In 2D, \citet{mjm2012} showed that the treatment of gravity
can be the difference between a failed and a successful explosion;
specifically, in some cases, the oft-used
``effective potential" approach \citep{mdj2006} leads to a dud, whereas
the more sophisticated (and more complicated to implement) \xcfc\ approach
leads to an explosion.
Additionally, a recent paper investigated the role of GR on the SASI
using \thornado\ \citep{dem2023} (in press).
They concluded that GR may have a significant impact on the SASI by drastically
lowering the rate at which the instability grows.
For these reasons, we advocate the field move to performing simulations
in full GR.

Contributions I have made to the field include investigating the role of GR
in the SASI (see Chapter~\ref{ch.sasi}) and developing the first DG solver
for the GR hydrodynamics equations under the xCFC
for application to CCSNe (see Chapter~\ref{ch.mp}).

The immediate outlook for this work is that, once the neutrino transport
has been incorporated, \thornado\ will be able to perform CCSN simulations
using DG methods.
It will be the first code to do so.
A slightly further outlook is that \thornado\ has the potential to perform
simulations of the post-merger phase of binary neutron star mergers
using spectral methods for the neutrino transport.

This document is laid out such that Chapter~\ref{ch.sasi} is an adaptation
of an accepted article on the effects of GR on the SASI,
Chapter~\ref{ch.mp} is an adaptation of an article in preparation on our
numerical methods for GR hydrodynamics,
and we close with a conclusion.

%------------------------------------------------------------------------------%
