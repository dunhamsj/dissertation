\chapter{Introduction}

Stars more massive than about eight Solar masses end their nuclear
fusion-driven phase of existence in explosive, spectacular events
known as core-collapse supernovae (CCSNe).
These events are sources of some of the most extraordinary and
exquisite objects in the Universe, ranging from the microscopic heavy elements
they synthesize and distribute
to the macroscopic neutron stars into which their
(mostly) iron cores often transmute.
In addition, they are unique laboratories in which we can probe areas of
physics that are otherwise difficult, if not impossible, to probe with
terrestrial-based experiments; e.g., properties of dense nuclear matter and the
ever-elusive neutrino.
During a CCSN, the progenitor star releases its immense gravitational energy
($\sim\!\!10^{46}\,\J$, or $\sim\!\!10^{53}\,\erg$),
of which a mere 1\% goes into
the spectacular luminous displays we observe, while the remaining 99\% goes
into the release of neutrinos and antineutrinos \citep{bw2017}.
The violent motion of the stellar fluid generates gravitational waves that
are expected to be detectable with current generation gravitational wave
observatories such as LIGO, VIGO, and KAGRA \citep{aaa2020a,aaa2020b}
and next-generation gravitational wave observatories such as
the Einstein Telescope \citep{mvb2020} and Cosmic Explorer \citep{eaa2021}.

In order to understand and appreciate the products of CCSNe, we need to
understand how they operate.
The following paragraph is a paraphrase from the review article \citet{m2005}.
The basic picture is that when a massive star
(ZAMS mass $M\gtrsim8\,\msun$, with $\msun$ the mass of the Sun)
runs out of fuel, gravity overcomes pressure
and the star begins to collapse.
As the iron-core collapses, it can be thought of as being composed of
two pieces: a subsonically
collapsing ``inner-core" and a supersonically collapsing ``outer-core".
The collapse of the inner-core is eventually stopped because the nuclei are
forced into such close proximity with each other that they undergo a
phase transition from individual nuclei to bulk nuclear matter
(this is the creation of a proto-neutron star (PNS)),
causing the pressure to increase drastically as a result of the repulsive
nature of the strong force.
At this point, the core is supported by a combination of the strong force and
neutron degeneracy pressure.
The still infalling outer-core collides suddenly with the newly formed PNS,
generating a pressure wave that develops into shock wave.
The shock wave propagates outward, ultimately driving the explosion.
It is intuitive that the shock wave simply propagates outward, disrupting
the star.
Unfortunately, as often happens, reality is not so simple.

As the shock propagates, it loses energy to the newly-created and liberated
electron-type neutrinos as well as to the dissociation of the
infalling material through which it passes,
so much so that the shock stalls around 200 km from the center
(e.g., see \citet{hm1981}).
It is clear that the shock is revived, because if it weren't then matter
would continue to accrete onto the star, eventually making the core massive
enough to overcome the strong force and the neutron degeneracy pressure
and cause the PNS to collapse
to a black hole \citep{bw1985}.
(This does happen under certain conditions, but they are not discussed here.)

It was discovered numerically that while the shock is stable to
perturbations in one-dimensional (1D) models,
it is unstable to non-radial perturbations
in two-dimensional (2D) models \citep{bmd2003}.
This purely hydrodynamical instability has since become known as the
standing accretion shock instability (SASI) and it very likely plays an
important role in the explosion mechanism for more massive progenitors.
This discovery also strongly implies that CCSN explosions are
inherently multi-dimensional.

One effect of the SASI is turbulence generated below the shock,
which, in addition to neutrino-driven convection,
can increase the amount of time that a fluid particle spends
in the ``gain region", a region below the shock in which there is a
net heating due to neutrinos \citep{co2015}.
Since the fluid spends more time in this region it can be heated more than if
turbulence and convection were absent.
This extra heating can help revive the shock.
Three-dimensional (3D) models are necessary for several reasons,
one of which is that turbulence is non-existent in 1D and in 2D it
exhibits an inverse energy-cascade, where energy is artificially transmitted
from small scales to large scales \citep{yem2017}.
Therefore, to truly study turbulence in CCSNe we need 3D simulations,
with no assumptions of symmetry.

Although exoergic fusion stops at iron,
CCSNe produce elements of much higher atomic mass number, $A$.
In fact, CCSNe are responsible for most of the elements in the Universe with
$16\lesssim A\lesssim90$ \citep{bw2017}.
Two products in particular, $^{56}$Ni and $^{57}$Ni, are critical for SN light
curves (their half-lives correlate well with the decline of the light curve),
and the amounts of these elements depend on the explosion mechanism.
The explosion energy is sufficient to expel the matter away from the star
and throughout the galaxy, enriching the interstellar medium with heavy
elements.
The distribution of metals in galaxies has implications for galactic
evolution, so an understanding of how the heavy elements are distributed,
and especially how much of each element is distributed,
is crucial for galaxy evolution models.

One of the elements produced and distributed is carbon,
the element on which life as we know it is based.
Therefore, there is additionally
a metaphysical reason to understand CCSNe:
In some sense, it is to them that we literally owe our entire existence!

The most obvious products of CCSNe are the copious amount of photons
released in the explosion process.
However, preceding the fireworks is something arguably much more interesting
and magnificent---the gravitational waves emitted from the asymmetry of the
mass-energy distribution.
These waves will be more like bursts, lasting less than one second,
but will be in the LIGO band and should be detectable by aLIGO
out to several kpc \citep{aaa2016}.
From full simulations of CCSNe, we obtain the components of the stress-energy
tensor, from which we can compute the mass-quadrupole moment,
which can tell us the waveforms of the gravitational waves emitted.

Another product of CCSNe is the plethora of neutrinos released.
As mentioned earlier, about 99\% of the energy in a CCSN
(amounting to approximately $0.15\,\msun\,c^{2}$) is carried away by neutrinos.
These weakly interacting particles are some of the most elusive particles known.
Their practically infinitesimal cross-section makes them extremely difficult
to detect.
A CCSN is one of the few phenomena that produces a sufficient amount of
neutrinos to be detectable on Earth.

\sd{Discuss importance of GR}
\sd{Mention current codes and the methods they use}

%------------------------------------------------------------------------------%
