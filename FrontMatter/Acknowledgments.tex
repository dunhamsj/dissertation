
%%%%%%%%%%%%%%%%%%%%%%%%%%%%%%%%%%%%%%%%%%%%%%%%%%%%%%%%%%%%%%%%%%%%%%%%%%%%
\chapter*{ACKNOWLEDGMENTS}
%\addcontentsline{toc}{chapter}{ACKNOWLEDGMENTS}
\vspace{7mm}

\begin{doublespace}

When I began to write this section, I quickly realized that to truly
acknowledge everyone that merits acknowledgement I would have to recount
my entire life story.
This is mostly due to determinism: where I am now is a complicated function
of every decision that I, and everyone else, has made.
Since my memory isn't good enough to recall every decision, I can only do my
best to thank all of those that have contributed to this dissertation,
both tangibly and intangibly.

In roughly chronological order, I would first like to thank Mr. Kapp,
my first college-level physics teacher, who
helped to keep my physics knowledge grounded in practicality.
Secondly, I would like to thank Mr. Quail, my calculus II teacher
who really let me know that I had
a natural aptitude for math, even at the college level.
Once I graduated from Washtenaw Community College and transferred to the
University of Michigan, professor Keren Sharon put me on the research track
by asking me if I would like to work with her.
I would not have had the confidence to ask a professor such a question,
so that was a pivotal moment for my career.
She went on to be my undergraduate advisor, and could not have been more
supportive of me and my future.
When I didn't get into any graduate schools, professor Nuria Calvet went out
of her way to mention me to Keivan Stassun, a professor at Vanderbilt
who was a co-founder of the Fisk-Vanderbilt Master's-to-PhD Bridge Program.
Another pivotal moment happened when I applied and was accepted into the
bridge program, and thanks to that goes to Nuria (and a little to me,
for being so great that I was accepted).
Once there, I met Kelly Holley--Bockelmann, a professor who would become the
co-chair of both my Master's and Ph.D. dissertation committees.
Again, I could not have asked for a better or more supportive person to be
(in some sense) in charge of my future.
Kelly introduced my to professor Anthony Mezzacappa,
who would also be on both of my
committees, and professor Eirik Endeve,
who would be co-chair of both of my committees.
Yet again, I was fortunate to have great and supportive people helping me to
navigate my future.
In case it isn't well known,
it is certainly not always the case to have a committee
loaded with people who truly have your best interests in mind.

Of course, a dissertation requires much more than academic guidance.
I also thank my wife, Kate, for her unyielding support throughout this journey.
Again, it is unfortunately not the case that one's spouse is always so
supportive, and I am very lucky that Kate has been.
I also thank my parents, Sandy and Steve, for their support and encouragement.
My life, like most people's, has been a bit of an emotional rollercoaster,
and they have always been there and have been supportive of my decisions.
I also want to thank my brother, Cal, to whom I have looked up since I was
very young and continue to do so today.
I thank my grandparents, Jack and Jean, who always knew the value
of education and who helped me begin my academic journey with a college fund.
I know they would be beyond proud to see how far I've come.

In addition to academic and family support, my journey has been greatly
helped by my friends and peers.
Moving to a new state is tough, and
getting through the graduate classes was much easier because of my
bridge cohort, Dax, Antonio, and Natasha, as well as other members of the
Vanderbilt astronomy group, Savannah, John, Karl, George, Laura, and Victor.
You all helped me get through this, and I am grateful to you all.
Thanks also to my childhood friends from Michigan,
Jeff, Evan, Travis, Kris, and Scott.
You've always been there for me and always made coming home extra special.

Lastly, I thank my pets, Lily, Daisy, Bo, and Penny.
They all gave and give so much joy and fun and enrich my life tremendously.

My sincerest apologies to anyone I left out, but please know that appreciate
you.

\end{doublespace}
