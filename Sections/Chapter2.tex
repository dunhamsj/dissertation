\clearpage % clear the prior chapter's page

\chapter{Title of Chapter: Related Work}\label{CH2_RelatedWork}
%\vspace{-7mm}
%\bigskip

\section{First Level Heading}
Begin each chapter at the top of a new page. Follow the chapter number and chapter title with the same amount of space (line and one-half, double space, or “two enters, with spacing set to double space”). Use this same amount of space to precede first -and second- level headings, and before and after figures and tables.

\subsection{Second Level Heading}
The number of levels and the placement of the headings and subheadings will vary, dependent on departmental requirements or preference. Headings may be centered, left justified, in bold face, italicized, indented or numbered. Use the same style throughout the document. Be consistent with spacing and heading styles.

\section{Another First Level Heading}
Begin the use of Arabic numbering on the first page of text. Continue consecutive Arabic page numbering throughout the remainder of the document, including the appendices and references.


\begin{table}[b]
\scriptsize
\renewcommand{\tabcolsep}{0.09cm}
\centering

\newcolumntype{Y}{>{\raggedright\arraybackslash}X}


\begin{tabularx}{0.6\textwidth}{YY} \\
\toprule \\
 Cats & Dogs  \\ \\
\midrule \\
Persian & German Shepherd  \\ 
Maine Coon & Bulldog  \\ 
Bengal & Poodle  \\ 
British Shorthair & Labrador Retriever  \\ 
Siamese & Golden Retriever  \\ 
Sphynx & French Bulldog  \\ 
Ragdoll & Great Dane  \\ 
Savannah & Dachshund  \\ 
Munchkin & Pomeranian  \\ 

\bottomrule \\
\end{tabularx} 





\caption{A sample table.}
\label{table:test}
\end{table}

\subsection{Second Level Heading}
For this template, Roman numbering of chapters was chosen, along with local (decimal) numbering of subheadings, tables, and figures. This is a very common approach, but not the only one. 

\subsubsection{Third Level Heading}
The dissertation is a long document, so there may be several nested levels for content. 

\paragraph{Fourth Level Heading}
More nesting, as well.

\paragraph{Another Fourth Level Heading}
Whatever you do, stay consistent.


