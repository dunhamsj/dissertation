% Generated by /home/kkadoogan/Work/Analysis/thornadoHydroXCFC_MethodsPaper_Software/./computeL1Error.py
% on 2024-02-03 17:43:15.562823
% Modified to work with Vanderbilt dissertation format
\begin{table}[b]
  \scriptsize
  \renewcommand{\tabcolsep}{0.09cm}
  \centering
  \begin{tabularx}{0.4\textwidth}{cccccc} \\
    \toprule \\
    $N$              &
    $\NK$            &
    Mesh             &
    Flux Corrections &
    $E_{\rho}$       \\ \\
    \midrule \\
2 & 32 & Single & N/A & $2.19\times10^{-5}$ \\
2 & 48 & Multi & Yes & $1.20\times10^{-5}$ \\
2 & 48 & Multi & No & $3.59\times10^{-4}$ \\
2 & 64 & Single & N/A & $5.13\times10^{-6}$ \\
3 & 32 & Single & N/A & $7.00\times10^{-6}$ \\
3 & 48 & Multi & Yes & $3.07\times10^{-6}$ \\
3 & 48 & Multi & No & $1.42\times10^{-5}$ \\
3 & 64 & Single & N/A & $9.77\times10^{-7}$ \\
  \bottomrule \\
  \end{tabularx}
  \caption{%
Effects of flux corrections on the accuracy
of our numerical method.
The first column is the number of DG nodes per element,
the second column is the number of elements,
the third column specifies whether or not the simulation used a single-
or multi-level mesh,
the fourth column specifies whether or not a multi-level mesh simulation
applied flux corrections,
and the fifth column denotes the error as defined in
\myeqref{eq.Error}.}
  \label{tab.CR}
\end{table}
