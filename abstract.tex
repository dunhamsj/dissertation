\documentclass[10pt]{article}
\usepackage[lmargin=1in,rmargin=1in]{geometry}
\usepackage{setspace}

\newcommand{\thornado}{\texttt{thornado}}

\newcommand\Fontvi{\fontsize{9}{9.2}\selectfont}

\begin{document}

\pagenumbering{gobble}

\begin{flushright}
\underline{ASTROPHYSICS}
\end{flushright}\vspace{1em}

\flushleft

\begin{center}
\thornado-Hydro, xCFC: A Discontinuous Galerkin Hydrodynamics
Solver for General Relativistic
Core-Collapse Supernova Simulations
\end{center}\vspace{1em}

\begin{center}
Samuel John Dunham
\end{center}\vspace{1em}

\centerline{\underline{Dissertation under the direction of
Professor Kelly Holley--Bockelmann and Professor Eirik Endeve}}
\vspace{1em}

\begin{doublespace}

Modeling core-collapse supernovae (CCSNe) has been an on-going effort
since the 1960s.
Despite advances in physics theory, physics experiments, and technology,
consistent, multi-dimensional CCSN explosions are still difficult to achieve,
and a consensus is yet to be reached regarding what physics is required.
Here, we present the status of a novel simulation code, \thornado,
which aims to simulate the neutrino-radiation kinetics
and the stellar hydrodynamics of
CCSNe using high-order discontinuous Galerkin methods
under the extended conformally flat condition (xCFC), an approximation to GR.
Specifically, we show the progress in implementing a
solver for the general relativistic (GR)
hydrodynamics equations and we show results from several test problems
designed to demonstrate the abilities and capabilities of the code.
This includes evolving a Kelvin--Helmholtz instability problem,
a multi-dimensional blast wave, and
the adiabatic collapse, bounce, and post-bounce phases
of a realistic progenitor.
These simulations are performed
in parallel with MPI and with adaptive mesh refinement via a coupling
to AMReX, a software framework designed for exascale systems.
This adiabatic collapse test also demonstrates the coupling of \thornado\ to
Poseidon, a hybrid spectral/finite-element xCFC metric solver
that is also designed
with an eye toward simulating CCSNe.
We also analyze the performance of the solver with both
strong- and weak-scaling tests.
As a science application of our code,
we show results from a study of the effects of GR on the
standing accretion shock instability, a hydrodynamical instability
that manifests in, among other scenarios, CCSNe.

\end{doublespace}
\vspace{3em}

\noindent Approved \rule[-3pt]{3.5in}{.5pt} \hskip 0.1in %
          Date     \rule[-3pt]{1.5in}{.5pt}
\hspace*{1.3in} Kelly Holley-Bockelmann, Ph.D. \\[0.1in]
\noindent Approved \rule[-3pt]{3.5in}{.5pt} \hskip 0.1in %
          Date     \rule[-3pt]{1.5in}{.5pt}
\hspace*{1.3in} Eirik Endeve, Ph.D.

\end{document}
